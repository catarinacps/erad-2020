% Intended LaTeX compiler: pdflatex
\documentclass[12pt]{article}
\usepackage[utf8]{inputenc}
\usepackage[T1]{fontenc}
\usepackage{graphicx}
\usepackage{grffile}
\usepackage{longtable}
\usepackage{wrapfig}
\usepackage{rotating}
\usepackage[normalem]{ulem}
\usepackage{amsmath}
\usepackage{textcomp}
\usepackage{amssymb}
\usepackage{capt-of}
\usepackage{hyperref}
\usepackage[cache=false,outputdir=org-exports]{minted}
\usepackage[T1]{fontenc}
\usepackage[utf8]{inputenx}
\usepackage{placeins}
\usepackage{sbc-template}
\usepackage[brazil, brazilian]{babel}
\usepackage[utf8]{inputenc}
\usepackage[T1]{fontenc}
\usepackage{graphicx}
\usepackage[caption=false]{subfig}
\usepackage{booktabs}
\usepackage{hyphenat}
\hyphenation{e-la-bo-ra-ção re-pre-sen-tar}
\date{}
\title{Paper for ERAD 2020}
\hypersetup{
 pdfauthor={Henrique},
 pdftitle={Paper for ERAD 2020},
 pdfkeywords={},
 pdfsubject={},
 pdfcreator={Emacs 26.3 (Org mode 9.3)},
 pdflang={Brazilian}}
\begin{document}

\author{
   Henrique Corrêa Pereira da Silva\and%\inst{1},
   Marcelo Cogo Milletto\and
   Vinicius Garcia Pinto\and
   Lucas Mello Schnorr
}

\address{
  Instituto de Informática -- Universidade Federal do Rio Grande do Sul (UFRGS)\\
  Caixa Postal 15.064 -- 91.501-970 -- Porto Alegre -- RS -- Brasil
  \email{\{hcpsilva,mcmiletto,vgpinto,schnorr\}@inf.ufrgs.br}
}

\maketitle

\begin{resumo}

\end{resumo}

\section{Introdução, Objetivos e Trabalhos Relacionados}
\label{sec:org5909530}

\section{Metodologia de Coleta e Análise de Dados}
\label{sec:org48b5dae}

\subsection{Manipulação de dados com ferramentas modernas de \emph{data science}}
\label{sec:orgc7ce29c}

\subsection{Projeto Experimental}
\label{sec:org1cafe92}

\section{Resultados Experimentais e Observações}
\label{sec:orga1829da}

\subsection{Configuração Experimental}
\label{sec:org130c3a6}

\subsection{Comparação do Escalonamento entre os três \emph{runtimes}}
\label{sec:org4f30eed}

\subsection{Análise de Ociosidade por \emph{Worker}}
\label{sec:orgc74de53}

\subsection{Diferenças de tempo de execução dos \emph{kernels} em função do \emph{runtime}}
\label{sec:orgb8632bb}

\section{Conclusão e Trabalhos Futuros}
\label{sec:org4cd723b}
\end{document}
